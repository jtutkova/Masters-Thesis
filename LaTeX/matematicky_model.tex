\chapter{Matematický model}
\label{kap:matematicky_model}
Upravme rovnice pre obálku sféry tak, aby zodpovedali obálke elipsoidov. Vezmime elipsoid so škálovaním v smere súradnicových osí konštantnými reálnymi číslami $a,b,c.$ Upravme škálovanie elipsoidu tak, aby $a$, škálovanie v smere osi $x$ zodpovedalo škálovaniu v dotykovom smere priestorovej krivky $m(t).$
Teda pre elipsoid s rovnicou
\begin{equation*}
\frac{{(x - m_1(t))^2}}{{a^2}} + \frac{{(y - m_2(t))^2}}{{b^2}} + \frac{{(z - m_3(t))^2}}{{c^2}}= 1
\end{equation*} 
zmeníme štandardnú bázu $\vec{e}_1, \vec{e}_2, \vec{e}_3$ na bázu Frenetovho repéra $\vec{t}, \vec{n}, \vec{b}$ v každom bode krivky $m(t).$ Naše očakávanie je, že obálku  elipsoidu škálovaného v dotykovom smere budeme môcť zostrojiť ako obálku kružníc, a teda, deriváciou jednoparametrického systému elipsoidov je podľa parametra $t$ bude opäť rovina. V ďalších výpočtoch budeme parameter $t$ pre väčšiu prehľadnosť zápisov vynechávať.
\section{Obálka elipsoidov}
Obálku sfér vieme zapísať skalárnym súčinom, no existuje však aj všeobecný zápis pre plochy druhého rádu, a to maticový 
$$
S: X^TM(t)X = 0.
$$
Všeobecné rovnice pre sféru v priestore sú
$ x^2 +y^2 +z^2 -2xm_1 -2ym_2 - 2zm_3 + m_1^2 + m_2^2 + m_3^2 - r^2 =0
,$
potom maticový zápis
$$
\left(\begin{matrix} x \\ y \\ z  \\ 1
\end{matrix} \right)^T \left(\begin{matrix} 
1 & 0 & 0 & - m_1 \\
0 & 1 & 0 & - m_2 \\
0 & 0 & 1 & - m_3 \\
- m_1 & - m_2 & - m_3 &  m_1^2 + m_2^2 + m_3^2 - r^2 \\
\end{matrix} \right)\left(\begin{matrix} x \\ y \\ z \\ 1
\end{matrix} \right) = 0. 
$$
Podobne zapíšme rovnice pre elipsoid $Q$
$$ \frac{x^2}{a^2} + \frac{y^2}{b^2} + \frac{z^2}{c^2} - 2 \frac{xm_1}{a^2} - 2 \frac{ym_2}{b^2} - 2 \frac{zm_3}{c^2} + \frac{m_1^2}{a^2} + \frac{m_2^2}{b^2} + \frac{m_3^2}{c^2} - 1 = 0,$$
$$
\left(\begin{matrix} x \\ y \\ z  \\ 1
\end{matrix} \right)^T \left(\begin{matrix} 
\frac{1}{a^2} & 0 & 0 & \frac{-m_1}{a^2} \\
0 & \frac{1}{b^2} & 0 & \frac{-m_2}{b^2} \\
0 & 0 & \frac{1}{c^2} & \frac{-m_3}{c^2} \\
\frac{-m_1}{a^2} & \frac{-m_2}{b^2} & \frac{-m_3}{c^2} & \frac{m_1}{a^2} + \frac{m_2}{b^2} + \frac{m_3}{c^2} - 1 \\
\end{matrix} \right)\left(\begin{matrix} x \\ y \\ z \\ 1
\end{matrix} \right) = 0,
$$
potom deriváciu elipsoidu
$$
-\frac{\dot{m}_1x}{a^2} + \frac{\dot{m}_2y}{b^2} + \frac{\dot{m}_3z}{c^2} + \frac{m_1\dot{m}_1}{a^2} + \frac{m_2 \dot{m}_2}{b^2} + \frac{m_3 \dot{m}_3}{c^2} = 0,
$$
v maticovom zápise 
$$
\left(\begin{matrix} x \\ y \\ z  \\ 1
\end{matrix} \right)^T \left(\begin{matrix} 
0 & 0 & 0 & \frac{-\dot{m}_1}{a^2} \\
0 & 0 & 0 & \frac{-\dot{m}_2}{b^2} \\
0 & 0 & 0 & \frac{-\dot{m}_3}{c^2} \\
\frac{-\dot{m}_1}{a^2} & \frac{-\dot{m}_2}{b^2} & \frac{-\dot{m}_3}{c^2} & \frac{2m_1\dot{m}_1}{a^2} + \frac{2m_2 \dot{m}_2}{b^2} + \frac{2m_3 \dot{m}_3}{c^2}\\
\end{matrix} \right)\left(\begin{matrix} x \\ y \\ z \\ 1
\end{matrix} \right) = 0,
$$
je rovina kolmá na dotykový vektor $\dot{m}(t), $ ak platí, že vektory $(\dot{m}_1, \dot{m}_2, \dot{m}_3) $ a $(\frac{\dot{m}_1}{a^2}, \frac{\dot{m}_2}{b^2}, \frac{\dot{m}_3}{c^2}) $ sú lineárne závislé. Ak budeme elipsoid škálovať tak, že $a, b, c$ budú škálovacie faktory také, aby sa elipsoid naškáloval v dotykovom smere, tak kolmá rovina na dotykový vektor vytne kružnicu, pretože elipsoid je v ostatných smeroch homogénny. Vyriešme toto odvodenie pre jednoduchší prípad.

\section{Obálka elíps}
\subsection{Zmena bázy}
Nech $m(t) \colon  I \subseteq \mathbb{R} \rightarrow \mathbb{R}^2$ je aspoň dvakrát diferencovateľná krivka, položme elipsu $Q$ s osami $a_0, b_0 \in \mathbb{R}$ na túto krivku v každom bode $t \in I.$
\begin{equation*}
\frac{{(x - m_1(t))^2}}{a_0^2} + \frac{{(y - m_2(t))^2}}{b_0^2} = 1
\end{equation*}
Matica prechodu k novej báze je tvaru
$$
A(t) = \left(\begin{matrix} \vec{t}(t) \quad \vec{n}(t)
\end{matrix} \right),
$$
čo po rozpísaní do súradníc $\vec{t}(t) = \frac{1}{ \| \dot{m}(t) \|}( \dot{m}_1(t),  \dot{m}_2(t))$ a $\vec{n}(t) = \frac{1}{ \| \dot{m}(t) \|}( -\dot{m}_2(t),  \dot{m}_1(t))$ dáva 
$$
A(t) = \frac{1}{ \| \dot{m}(t) \|} \left(\begin{matrix}
   \dot{m}_1(t) & n_1(t) \\
   \dot{m}_2(t) & n_2(t)
\end{matrix} \right)
$$
Keďže sme maticu $A(t)$ zostrojili tak, aby bola ortogonálna, ľahko dostávame aj vyjadrenie inverznej matice
$$
A^{-1}(t) = A^{T}(t) = \frac{1}{ \| \dot{m}(t) \|} \left(\begin{matrix}
  \dot{m}_1(t) & \dot{m}_2(t) \\
    n_1(t) & n_2(t)
\end{matrix}\right)
$$
Vyjadrime elipsu $Q$ v novej báze nasledovnou transformáciou
\begin{align*}
\frac{u^2(t)}{a^2} + \frac{v^2(t)}{b^2} = 1,
\end{align*}
kde 
$$
\left(\begin{matrix}
u(t) \\
v(t)
\end{matrix}\right) = \frac{1}{ \| \dot{m}(t) \|}
\left(\begin{matrix}
  \dot{m}_1(t) & \dot{m}_2(t) \\
    n_1(t) & n_2(t)
\end{matrix}\right)
\left(\begin{matrix}
x-m_1(t) \\
y-m_2(t) \\
\end{matrix}\right)
$$
Elipsa $Q$ sa potom transformuje na 
\begin{equation} 
\label{eq:elipsa_v_novej_baze}
\frac{1}{\|\dot{m}\|^2} \left( (x - m_1)^2 \left( \frac{{\dot{m}_1}^2}{a^2} + \frac{{\dot{m}_2}^2}{b^2} \right) + 2(x - m_1)(y - m_2)\dot{m}_1\dot{m}_2 \left( \frac{1}{a^2} - \frac{1}{b^2} \right) \\
+ (y - m_2)^2 \left( \frac{{\dot{m}_2}^2}{a^2} + \frac{{\dot{m}_1}^2}{b^2} \right) \right) - 1 = 0.
\end{equation}
Po vynásobení rovnice štvorcom normy $\| \dot{m}\|^2$ a členom $a^2b^2$ sa výraz zjednoduší
\begin{align*}
&(x-m_1)^2(b^2 \dot{m}_1^2 + a^2 \dot{m}_2^2) + 2(x-m_1)(y-m_2)\dot{m}_1\dot{m}_2(b^2-a^2) \\
&+(y-m_2)^2(b^2 \dot{m}_2^2 + a^2 \dot{m}_1^2) - a^2 b^2 \| \dot{m}\|^2 = 0,
\end{align*}
kde konštanta $a$ zabezpečí škálovanie elipsy v dotykovom smere a konštanta $b$ určí škálovanie v normálovom smere. Výber orientácie normály nemá na výsledok žiaden vplyv. Štandardne uvažujeme $\vec{n}=(-\dot{m}_2, \dot{m}_1).$

\begin{example}[Parabola]
Majme parabolu s parametrizáciou $m(t)=(t, t^2), $ kde $\dot{m}(t)=(1, 2t),$ Transformujme elipsu v stredovom zápise so škálovaním v smere súradnicových osí
\begin{equation*}
\frac{(x - t)^2}{a_0^2} + \frac{(y - t^2)^2}{b_0^2} = 1
\end{equation*}
na elipsu so škálovaním v dotykovom a normálovom smere
\begin{align*}
(&x-t)^2(b^2 + 4a^2t^2) + 2(x-t)(y-t^2)2t(b^2-4a^2t^2)+(y-t^2)^2(4b^2t^2 + a^2) \\
 - &a^2b^2(1+4t^2) = 0.
\end{align*} 
\end{example}

\subsection{Výpočet obálky elíps}
Prepíšme rovnicu \ref{eq:elipsa_v_novej_baze} do maticového zápisu. Máme kužeľosečku $Q$ tvaru $$
Ax^2 + Bxy + Cy^2 + Dx + Ey + F = 0,$$
ktorá má v maticovom tvare vyjadrenie $Q: X^TM(t)X = 0$
$$
\left(\begin{matrix} x \\ y \\ z  \\ 1
\end{matrix} \right)^T \left(\begin{matrix} 
A & \frac{B}{2} & \frac{D}{2} \\
\frac{B}{2} & C & \frac{E}{2} \\
\frac{D}{2} & \frac{E}{2} & F 
\end{matrix} \right)\left(\begin{matrix} x \\ y \\ z \\ 1
\end{matrix} \right) = 0,
$$ 
Všetky prvky matice $M(t)$ sú závislé od $t$.

Jej koeficienty po vynásobení normou $ \| \dot{m} \|^2$ sú tvaru
\begin{align*}
A &= \frac{{\dot{m}_1}^2}{a^2} + \frac{{\dot{m}_2}^2}{b^2} \\
\frac{B}{2} &= \dot{m}_1 \dot{m}_2 \left( \frac{1}{a^2} - \frac{1}{b^2} \right) \\
C &= \frac{{\dot{m}_2}^2}{a^2} + \frac{{\dot{m}_1}^2}{b^2} \\
\frac{D}{2} &= - m_1 \left( \frac{{\dot{m}_1}^2}{a^2} + \frac{{\dot{m}_2}^2}{b^2} \right) - m_2 \dot{m}_1 \dot{m}_2 \left(\frac{1}{a^2} - \frac{1}{b^2}\right) \\
\frac{E}{2} &= - m_2 \left( \frac{{\dot{m}_2}^2}{a^2} + \frac{{\dot{m}_1}^2}{b^2} \right) - m_1 \dot{m}_1 \dot{m}_2 \left( \frac{1}{a^2} - \frac{1}{b^2} \right) \\
F &= m_1^2 \left( \frac{{\dot{m}_1}^2}{a^2} + \frac{{\dot{m}_2}^2}{b^2} \right) + 2 m_1 m_2 \dot{m}_1 \dot{m}_2 \left( \frac{1}{a^2} - \frac{1}{b^2} \right) + m_2^2 \left( \frac{{\dot{m}_2}^2}{a^2} + \frac{{\dot{m}_1}^2}{b^2} \right) -  \| \dot{m}  \|^2 \\
\end{align*}
Determinant matice $M(t)$ je $ - \frac{1}{a^2b^2} < 0.$ Matica reprezentuje elipsu.

Pozrime sa deriváciu jednoparametrického systému elíps podľa parametra $t,$ označme $\dot{Q}$. Máme teda kužeľosečku $\dot{Q}$ tvaru 
$$
\dot{A}x^2 + \dot{B}xy + \dot{C}y^2 + \dot{D}x + \dot{E}y + \dot{F} = 0$$
ktorej priradíme maticu  $\dot{M}(t).$

\begin{align*}
\dot{A} &= 2 \left( \frac{\dot{m}_1 \ddot{m}_1}{a^2} + \frac{\dot{m}_2 \ddot{m}_2}{b^2} \right) \\
\frac{\dot{B}}{2} &= 2 (\ddot{m}_1 \dot{m}_2 + \dot{m}_1 \ddot{m}_2) \left( \frac{1}{a^2} - \frac{1}{b^2} \right) \\
\dot{C} &= 2 \left( \frac{\dot{m}_2 \ddot{m}_2}{a^2} + \frac{\dot{m}_1 \ddot{m}_1}{b^2} \right) \\
\frac{\dot{D}}{2} &= - 2\dot{m_1} \left( \frac{\dot{m}_1^2}{a^2} + \frac{\dot{m}_2^2}{b^2} \right) - 4m_1 \left( \frac{\dot{m}_1 \ddot{m}_1}{a^2} + \frac{\dot{m}_2 \ddot{m}_2 }{b^2} \right) -2 \left( \frac{1}{a^2} - \frac{1}{b^2} \right) ( \dot{m}_1 \dot{m}_2^2 +  m_2 \dot{m}_1 \ddot{m}_1 + m_2 \dot{m}_1 \ddot{m}_2)  \\
\frac{\dot{E}}{2} &= - 2\dot{m_2} \left( \frac{\dot{m}_2^2}{a^2} + \frac{\dot{m}_1^2}{b^2} \right) - 4m_2 \left( \frac{\dot{m}_2 \ddot{m}_2}{a^2} + \frac{\dot{m}_1 \ddot{m}_1 }{b^2} \right) -2 \left( \frac{1}{a^2} - \frac{1}{b^2} \right) ( \dot{m}_1^2  \dot{m}_2 + m_1 \dot{m}_2 \ddot{m}_1  + m_1 \dot{m}_1 \ddot{m}_2)  \\
\dot{F} &= 2 \left( m_1 \dot{m_1} \left( \frac{\dot{m}_1^2}{a^2} + \frac{\dot{m}_2^2}{b^2} \right) + m_1^2 \left( \frac{\dot{m}_1 \ddot{m}_1}{a^2} + \frac{\dot{m}_2 \ddot{m}_2 }{b^2} \right) + \left( \frac{1}{a^2} - \frac{1}{b^2} \right) ( m_2 \dot{m}_1^2  \dot{m}_2 + m_1 \dot{m}_1 \dot{m}_2^2 \\
+ m_1 m_2 \dot{m}_1 \ddot{m}_2 + m_1 m_2 \dot{m}_2 \ddot{m}_1) + m_2 \dot{m_2} \left( \frac{\dot{m}_2^2}{a^2} + \frac{\dot{m}_1^2}{b^2} \right) + m_2^2 \left( \frac{\dot{m}_2 \ddot{m}_2}{a^2} + \frac{\dot{m}_1 \ddot{m}_1}{b^2} \right) \right) \\
\end{align*}

Determinant matice $\dot{M}(t) = 0, $ čo znamená, že kužeľosečka je singulárna.
Pozrime sa na hodnotu subdeterminantu $\dot{M}_{33}(t) = \dot{A} \dot{C} - \frac{\dot{B}^2}{4} < 0.$ To znamená, že kužeľosečka sa degeneruje na dve rôznobežné priamky.

