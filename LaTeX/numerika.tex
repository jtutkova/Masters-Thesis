\chapter{Numerické metódy}
Pomocou diferenciálnych rovníc sa opisujú rôzne fyzikálne deje. Niektoré z nich sa riešia len obtiažne a niektoré vyriešiť analyticky nemožno. Preto sa na hľadanie riešenia používajú približné metódy, z ktorých niektoré využijeme. 

Pre účely hľadania obálky budeme potrebovať metódy pre riešenie jednej diferenciálnej rovnice prvého rádu so zadanou počiatočnou podmienkou. 

Spoločným znakom všetkých numerických metód je, že riešenie nehľadáme ako spojitú funkciu, definovanú na celom skúmanom intervale $[a, b] = J$, ale  hodnoty približného riešenia počítame v konečnom počte bodov $ a = x_0 < x_1 < \dots < x_n = b.$ Týmto bodom hovoríme uzlové body alebo uzly siete a množine $\{x_0, x_1, \dots, x_n\}$ hovoríme sieť. Rozdiel $h_i = x_{i+1} - x_{i}$ sa nazýva krok siete v uzle $x_i.$ Sieť je pravidelná (ekvidištantná), ak je krok $h$ medzi jednotlivými uzlami konštantný. Ak chceme poznať približnú hodnotu riešenia v inom ako uzlovom bode, môžeme použiť interpolačnú metódu, napr. nahradiť riešenie lomenou čiarou alebo splajnom prechádzajúcou vypočítanými bodmi \cite{Bab00}, \cite{Hla}.

Formulujme problém hľadania obálky nasledovne:

Hľadáme krivku $\gamma(t) \colon J \subseteq I \rightarrow \mathbb{E}^{2} $ takú, že 
\begin{enumerate}
\item $F(\gamma(t), t) = 0, $ 
\item $\langle \nabla F(\gamma(t), t), \dot{\gamma}(t) \rangle = 0.$
\end{enumerate}
Nech $\gamma(t) = (\gamma_{1}(t), \gamma_{2}(t)), $ potom nasledujúce podmienky možno prepísať ako
\begin{enumerate}
\item $F(\gamma_{1}(t), \gamma_{2}(t), t) = 0, $ 
\item $\dfrac{\partial F}{\partial x} (\gamma_1 (t), \gamma_2 (t), t) \dot{\gamma_{1}}(t) + \dfrac{\partial F}{\partial y}(\gamma_1 (t), \gamma_2 (t), t) \dot{\gamma_{2}}(t) = 0.$
\end{enumerate}
Poznáme začiatočný bod $ \gamma (t_0) =(x_0, y_0, t_0) $ a gradient v tomto bode $ \nabla F (\gamma (t_0), t_{0})$. Keďže úlohu riešime numericky, hľadáme na krivke $\gamma(t)$ bod $\gamma(t_1)$ taký, že
\begin{enumerate}
\item $F(\gamma(t_{1}), t_{1}) = 0, $ 
\item $\langle \nabla F(\gamma(t_{1}), t_{1}), \dot{\gamma}(t_{1}) \rangle = 0.$
\end{enumerate}
Aplikujme na tento problém známe numerické metódy.

\section{Eulerova metóda}
Majme danú začiatočnú úlohu $y'=f(x,y),$  $y(x_0)=y_0$ s podmienkami, ktoré zaisťujú existenciu a jednoznačnosť riešenia. Označme uzly $x_i = x_{0} + ih,$ kde $h > 0$ označuje dĺžku kroku. Vo všetkých bodoch siete podľa predošlej rovnice platí $$y'(x_i)=f(x_i,y(x_i)).$$
Deriváciu na ľavej strane rovnice môžeme nahradiť diferenciou, a tak dostávame
$$\dfrac{y(x_{i+1}) - y(x_i)}{h} = f(x_i, y(x_i)).$$
Ak nahradíme $y(x_i)$ približnou hodnotou $y_i,$ môžeme odtiaľ vyjadriť približnú hodnotu $y(x_{i+1})$ tak, že
$$y_{i+1}= y_{i}+ hf(x_i, y_i) \text{, pre } i=0,1,\dots, n.$$

Pomocou tohto vzorca vypočítame približnú hodnotu riešenia v ďalšom uzlovom bode pomocou hodnoty v predchádzajúcom uzle. Hodnotu riešenia v bode $x_0$ poznáme zo začiatočnej podmienky.

\subsection{Geometrická interpretácia Eulerovej metódy}
Pre vysvetlenie geometrickej interpretácie Eulerovej metódy najprv pripomeňme, že diferenciálnou rovnicou $y'=f(x,y)$ je dané tzv. smerové pole. V každom bode roviny $(x,y),$ ktorým prechádza nejaké riešenie tejto rovnice, je hodnota $f(x,y)$ rovná smernici dotyčnice ku grafu tohto riešenia. Preto si smerové pole môžeme predstaviť tak, že v každom bode roviny $(x,y)$ stojí šípka, ktorá ukazuje, ktorým smerom máme pokračovať, ak sa dostaneme do daného bodu.

Pri riešení diferenciálnej rovnice Eulerovou metódou postupujeme takto:
Vyjdeme z bodu $(x_0, y_0)$ smerom, ktorý udáva šípka v tomto bode, to znamená, po priamke s rovnicou $$y = y_0 + f(x_0, y_0)(x-x_0),$$ až kým nedôjdeme do bodu s x-ovou súradnicou $x_1.$ Ypsilonová súradnica tohoto bodu je $y_1 = y_0 + f(x_0, y_0)(x-x_0)=y_0+hf(x_0,y_0).$ Z bodu $(x_1,y_1)$ pokračujeme vo smere danom smerovým poľom v tomto bode, t. j. po priamke $$y = y_1 + f(x_1, y_1)(x-x_1),$$ kým nedôjdeme do bodu s x-ovou súradnicou $x_2$. Teda môžeme vypočítať približnú hodnotu riešenia v ďalšom uzlovom bode pomocou riešenia v predchádzajúcom uzlovom bode. 

Existujú metódy, v ktorých postupujeme dômyselnejšie ako v Eulerovej, no stále využívajú informácie iba z jedného predchádzajúceho kroku. Takéto metódy sa nazývajú jednokrokové. Metódy využívajúce informácie z niekoľkých predošlých krokov nazývame viackrokové.

Je celkom zrejmé, že nakoľko sa priblížime k presnému riešeniu, závisí na dĺžke kroku $h$, ktorú použijeme. Základnú vlastnosť, ktorú od použiteľnej numerickej metódy požadujeme, je aby riešenie získane metódou pre $h \mapsto 0$ konvergovalo k presnému riešeniu danej úlohy.
 
\subsection{Prispôsobenie Eulerovej metódy na výpočet obálky}
Ak poznáme začiatočný bod $(x_0,y_0,t_0)$, v tomto bode vieme vyčísliť gradient $\nabla F(x_0, y_0, t_0)$ implicitne zadaného systému kriviek $\mathcal{F}_t$. Zo začiatočného bodu sa pohneme v smere dotykového vektora $\vec{m}$ krokom $h$ ku hľadanej obálke. Vektor $\vec{m}$ vypočítame ako kolmý vektor ku gradientu $\nabla F(x_0, y_0, t_0)$ voľbou vhodnej orientácie  
$$\vec{m} = \perp \dfrac{\nabla F(x_0, y_0, t_0)}{\| \nabla F(x_0, y_0, t_0)\|}.$$ 
Teda, na rozdiel od využitia smerového poľa, poznáme v každom bode pole gradientné, no zachovávame princíp Eulerovej metódy. Nový bod obálky odhadneme ako
\begin{align*}
x_1 &= x_0 + h \vec{m}_{1}, \\
y_1 &= y_0 + h \vec{m}_{2}. \\
\end{align*} 
%$$ \text{teda }(x_1, y_1) = (x_0, y_0)+ h \vec{m}. $$
Následne dopočítame k novému bodu obálky $(x_1, y_1)$ prislúchajúci parameter $t_1$, ktorý zodpovedá krivke systému, na ktorej leží nový bod implicitnej rovnice $$F(x_1, y_1, t_1) = 0.$$
Výpočet pokračuje, kým parameter $t_i \leq t_{max},$ teda máme iteračnú metódu, kde
\begin{align*}
(x_{i+1}, y_{i+1}) &= (x_i, y_i)+ h \vec{m}, \\
F(x_{i+1}, y_{i+1}, t_{i+1}) &= 0,
\end{align*} 
kde $i$ patrí do indexovej množiny $M = \{ i \mid t_{i} \in [t_{min},t_{max}] \}.$

\subsection{Modifikácia Eulerovej metódy}
Ako už napovedá názov, budeme postupovať podobne ako v Eulerovej metóde. Najprv vypočítame pomocné hodnoty \(k_1\) a \(k_2\) a pomocou nich približnú hodnotu riešenia v ďalšom uzlovom bode.

V prvej modifikácii Eulerovej metódy počítame podľa vzorcov
\begin{align*}
k_1 &= f(x_n, y_n) \\
k_2 &= f\left(x_n + \frac{1}{2}h, y_n + \frac{1}{2}hk_1\right) \\
y_{n+1} &= y_n + hk_2,
\end{align*}
v druhej modifikácii podľa vzorcov
\begin{align*}
k_1 &= f(x_n, y_n) \\
k_2 &= f(x_n + h, y_n + hk_1) \\
y_{n+1} &= y_n + \frac{1}{2}h(k_1 + k_2).
\end{align*}
Obe modifikované Eulerove metódy sú druhého rádu. Geometricky sa tieto metódy interpretujú podobne ako Eulerova metóda. Na obrázkoch vidíme jeden krok prvej a druhej modifikovanej Eulerovej metódy.

V prvej modifikácii najprv nájdeme pomocný bod $P$, a to tak, že z bodu \([x_n, y_n]\) vyjdeme po priamke so smernicou \(f(x_n, y_n)\), t.j. rovnako ako pri Eulerovej metóde, ale dôjdeme iba do bodu s \(x\)-ovou súradnicou \(x_n + \frac{h}{2}\). Približnú hodnotu riešenia v bode \(x_{n+1}\) potom získame tak, že z bodu \([x_n, y_n]\) ideme po priamke so smernicou určenou smerovým poľom v bode \(P\), kým nedôjdeme do bodu s \(x\)-ovou súradnicou \(x_{n+1}\).

V druhej modifikácii skonštruujeme dva pomocné body \(P_1\) a \(P_2\). Bod \(P_1\) dostaneme jedným krokom Eulerovej metódy. Bod \(P_2\) potom získame tak, že z bodu \([x_n, y_n]\) ideme po priamke so smernicou danou smerovým poľom v bode \(P_1\) do bodu s \(x\)-ovou súradnicou \(x_{n+1}\). Nový bod \([x_{n+1}, y_{n+1}]\) potom leží v strede úsečky \(P_1P_2\).

\section{Runge-Kuttova metóda}
Runge-Kuttove metódy sú zo skupiny jednokrokových metód. Pri týchto metódach získame niekoľko odhadov derivácie $k_i$ v rôznych bodoch. Pre celý krok potom použijeme vážený priemer týchto derivácií s váhami $w_i$. Všeobecný tvar Runge-Kuttovej metódy je
\[ y_{n+1} = y_n + h(w_1k_1 + \ldots + w_mk_m), \]
kde
\[ k_1 = f(x_n, y_n), \]
\[ k_i = f(x_n + \alpha_ih, y_n + h \sum_{j=1}^{i-1} \beta_{ij}k_j), \quad i = 2, \ldots, m. \]
a \(w_i\), \(\alpha_i\), a \(\beta_{ij}\) sú konštanty volené tak, aby metóda mala maximálny rád.

V modifikovanej Eulerovej metóde to bolo \(w_1 = 0\), \(w_2 = 1\), \(\alpha_2 = \frac{1}{2}\) a \(\beta_{21} = \frac{1}{2}\); v druhej modifikácii \(w_1 = w_2 = \frac{1}{2}\), \(\alpha_2 = 1\) a \(\beta_{21} = 1\).

V praxi sa najviac používa následujúca metóda Runge-Kutta 4. rádu.
\[ y_{n+1} = y_n + \frac{1}{6}h(k_1 + 2k_2 + 2k_3 + k_4), \]
kde
\begin{align*}
k_1 &= f(x_n, y_n), \\
k_2 &= f(x_n + \frac{1}{2}h, y_n + \frac{1}{2}hk_1), \\
k_3 &= f(x_n + \frac{1}{2}h, y_n + \frac{1}{2}hk_2), \\
k_4 &= f(x_n + h, y_n + hk_3). \\
\end{align*}